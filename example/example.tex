\documentclass[polish,engineering]{wizthesis}

\usepackage{blindtext} % Lorem ipsum filler text

% Set up the thesis
\author{Imię i nazwisko dyplomanta}
\title{Tytuł pracy dyplomowej}
\supervisor{prof. dr hab. Jan Kowalski}
\fieldofstudy{Informatyka}
\keywords{tutaj podajemy najważniejsze słowa kluczowe (łącznie nie powinny być dłuższe niż 150 znaków).}
\summary{
Tutaj piszemy krótkie streszczenie pracy (nie powinno być dłuższe niż 530 znaków).
Sed et risus lorem. Etiam faucibus, nisi eget tempor accumsan, magna nunc pharetra lacus,
nec semper neque ligula at magna. Curabitur ultrices a tellus nec viverra. Nam vestibulum,
dui et ultricies convallis, augue leo vulputate dolor, ut varius ligula nisi eget metus.
In vestibulum nibh quis nulla tempus, gravida ornare velit pulvinar.
Nunc rhoncus nisi in turpis dignissim venenatis. Nunc sapien.
}

% Set up the style of code listings, optional
\setminted{frame=single,breaklines,linenos}




\begin{document}


\frontmatter % Disable page and chapter numbering


\maketitle

% \chapter* removes both abstracts from the table of contents
\chapter*{Streszczenie}

W tym miejscu umieszczamy streszczenie w języku polskim. \blindtext

{\let\clearpage\relax\chapter*{Abstract}} % Keep the english abstract on the same page

And here's the abstract in English. \blindtext


\tableofcontents


\chapter{Wstęp}

We wstępie zapowiadamy, o czym będzie praca. Próbujemy zachęcić czytelnika do dalszej lektury, np. krótko informując, dlaczego wybraliśmy właśnie ten temat i co nas w nim zainteresowało. \blindtext[2] \par \blindtext


\mainmatter % Re-enable page and chapter numbering


\chapter{Rozdział pierwszy}

Tabela \ref{tab:przykladowa} przedstawia przykładową tabelę. Do tworzenia tabeli służą m.in. środowiska \texttt{tabular} oraz \texttt{table}. Istnieje możliwość numeracji dwustopniowej, gdzie pierwsza cyfra oznacza numer rozdziału, a druga – kolejny numer tabeli w tym rozdziale. Tytuł powinien znajdować się centralnie nad tabelą, $12$ pkt odstępu od tekstu zasadniczego nad i pod tabelą wraz z tytułem. Jeśli tabela jest cytowana – należy podać centralnie pod tabelą źródło jej pochodzenia, np. opracowanie własne, opracowano na podstawie danych z GUS.
\begin{table}[ht]
  \caption{Podstawowa tabela}
  \centering
  \begin{tabular}{ccc}
    \hline
    \hline
    Państwo & PKB (w milionach USD) & Stopa bezrobocia \\ [0.5ex]
    \hline
    Stany Zjednoczone & 75 278 049 & 4,60\% \\
    Chiny             & 11 218 281 & 4,10\% \\
    Japonia           & 4 938 644  & 3,10\% \\
    Niemcy            & 3 466 639  & 6,00\% \\
    Wielka Brytania   & 2 629 188  & 4,60\% \\ [1ex]
    \hline
  \end{tabular}
  \caption*{\textit{Źródło: opracowanie własne}}
  \label{tab:przykladowa} 
\end{table}

Do cytowania używamy komendy \texttt{cite}. W nawiasie klamrowym podajemy klucz, którego użyliśmy w pliku \emph{bibliografia.bib}. Przykład: \cite{einstein} lub \cite[chap. 2]{latexcompanion}.


\section{Podrozdział pierwszy}

\begin{table}[ht]
  \caption{Podstawowa tabela}
  \centering
  \begin{tabular}{ccc}
    \hline
    \hline
    Państwo & PKB (w milionach USD) & Stopa bezrobocia \\ [0.5ex]
    \hline
    Stany Zjednoczone & 75 278 049 & 4,60\% \\
    Chiny             & 11 218 281 & 4,10\% \\
    Japonia           & 4 938 644  & 3,10\% \\
    Niemcy            & 3 466 639  & 6,00\% \\
    Wielka Brytania   & 2 629 188  & 4,60\% \\ [1ex]
    \hline
  \end{tabular}
  \caption*{\textit{Źródło: opracowanie własne}}
  \label{tab:przykladowa2} 
\end{table}

No to może równanie?

\begin{equation}
  E = mc^2
\end{equation}


\section{Podrozdział drugi}

Rysunki do pracy dyplomowej należy wstawiać w sposób podobny do wstawiania tabel, z zasadniczą różnicą polegającą na tym, że podpis powinno umieszczać się centralnie pod rysunkiem, a nie powyżej niego. Numeracja i sposób cytowania pozostają bez zmian, przy czym tabele i rysunki nie mają numeracji wspólnej, np. po Tabeli \ref{tab:przykladowa2} występuje Rysunek \ref{rys:przykladowy} (o ile jest to pierwszy rysunek rozdziału pierwszego), a nie Rysunek $1.3$.
\begin{figure}[ht]
  \centering                  
  \includegraphics[width=0.4\textwidth]{img/iz_logo.png}
  \caption{Podstawowy rysunek}
  \label{rys:przykladowy}
\end{figure}

W moim kodzie zrobiłem coś niesamowitego.
\begin{listing}[ht]
  \begin{minted}{c}
#include <stdio.h>

int main(void)
{
  printf("Hello World\n");
  return 0;
}
  \end{minted}
  \caption{Niesamowity kod}
  \label{lst:listing}
\end{listing}

\Blindtext


{\backmatter\chapter{Zakończenie}} % Disable this chapter number

W zakończeniu co udało nam się zrobić w pracy, a czasem także o tym, czego nie udało się zrobić. \blindtext[2]


\bibliographystyle{acm}
\bibliography{bibliography.bib}

\listoffigures

\listoftables

\listoflistings

\appendixpage


\appendix

\chapter{Załącznik}

Dodatek w pracach matematycznych również nie jest wymagany. Można w nim przedstawić np. jakiś dłuższy dowód, który z pewnych przyczyn pominęliśmy we właściwej części pracy lub (np. w przypadku prac statystycznych) umieścić dane, które analizowaliśmy.

\end{document}
